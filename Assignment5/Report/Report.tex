\documentclass[12pt]{extarticle}
\usepackage{amsmath}
\usepackage{amsfonts}
\usepackage{amssymb}
\usepackage{graphicx}
\usepackage{tikz}
\usepackage{array}
\usepackage{booktabs}
\usepackage{hyperref}
\usepackage{geometry}
\geometry{margin=1.5cm}
\usepackage{xepersian}
\settextfont[Path="./", Extension=".ttf"]{XB-Niloofar}
\setdigitfont[Path="./", Extension=".ttf"]{XB-Niloofar}

\begin{document}
	
	\title{تکلیف پنجم - آزمایش \lr{N400}}
	\author{مجمدمهدب شریف بیگی}
	\date{}
	\maketitle
	
	\section*{مقدمه}
	شما یک شرکت‌کننده از آزمایش \lr{N400} (فایل \lr{12\_N400\_preprocessed.set}) را همراه با فایل توصیف‌گر \lr{bin} مربوطه (\lr{BDF\_N400.txt}) دارید. لطفاً مراحل زیر را کامل کنید، از هر مرحله اسکرین‌شات بگیرید و آنها را در یک گزارش \lr{PDF} واحد گردآوری کنید.
	
	\section{ایجاد لیست رویداد کدهای رویداد}
	لیست رویداد کدهای رویداد ایجاد شده و به عنوان \lr{events.txt} ذخیره شد.
	
	\section{یافتن زمان پاسخ برای ردیف ۵ (کد رویداد ۲۰۱)}
	\textbf{راهنما:} زمان پاسخ = اختلاف زمانی بین کلمه هدف و رویداد پاسخ.
	
	\textbf{زمان پاسخ:} $747.07$ میلی‌ثانیه
	
	\section{اختصاص رویدادها به \lr{bin}ها}
	رویدادها با استفاده از فایل \lr{BDF\_N400.txt} به \lr{bin}ها اختصاص داده شدند.
	
	\subsection{تعداد رویدادها در هر \lr{bin}}
	\begin{itemize}
		\item \textbf{\lr{bin 1}:} ۶۰ رویداد (کلمه آغازگر، مرتبط با کلمه هدف بعدی)
		\item \textbf{\lr{bin 2}:} ۶۰ رویداد (کلمه آغازگر، نامرتبط با کلمه هدف بعدی)
		\item \textbf{\lr{bin 3}:} ۵۸ رویداد (کلمه هدف، مرتبط با آغازگر قبلی، به دنبال آن پاسخ صحیح)
		\item \textbf{\lr{bin 4}:} ۵۵ رویداد (کلمه هدف، نامرتبط با آغازگر قبلی، به دنبال آن پاسخ صحیح)
		\item \textbf{جمع:} ۲۳۳ رویداد در تمام \lr{bin}ها (از ۳۶۳ رویداد کل در فایل)
	\end{itemize}
	
	\section{دوره‌بندی داده‌ها و اعمال تصحیح خط پایه}
	\begin{itemize}
		\item \textbf{محدوده دوره:} ۲۰۰- تا ۸۰۰ میلی‌ثانیه
		\item \textbf{خط پایه:} فاصله قبل از محرک (۲۰۰- تا ۰ میلی‌ثانیه)
	\end{itemize}
	
	\section{محاسبه میانگین \lr{ERP}ها}
	میانگین \lr{ERP}ها (میانگین کلی در دوره‌ها) محاسبه شدند.
	
	\subsection{رسم \lr{ERP}ها}
	\subsubsection{\lr{bin}های ۱ و ۲ با هم}
	
	\begin{figure}[h]
		\centering
		\includegraphics[width=0.9\textwidth]{plot12.jpg}
		\caption{نمودار \lr{ERP}ها برای \lr{bin}های ۱ و ۲}
	\end{figure}
	
	\subsubsection{\lr{bin}های ۳ و ۴ با هم}
	
	\begin{figure}[h]
		\centering
		\includegraphics[width=0.9\textwidth]{plot34.jpg}
		\caption{نمودار \lr{ERP}ها برای \lr{bin}های ۳ و ۴}
	\end{figure}
	
	\section{ایجاد \lr{bin} اضافی (\lr{bin 5})}
	\lr{bin 5} به صورت \lr{bin 4} منهای \lr{bin 3} تعریف شد.
	
	\subsection{رسم \lr{bin 5}}
	
	\begin{figure}[h]
		\centering
		\includegraphics[width=0.9\textwidth]{plot5.jpg}
		\caption{نمودار \lr{ERP} برای \lr{bin 5} (موج تفاضلی)}
	\end{figure}
	
	\subsection{توضیح اطلاعات قابل استنتاج از موج تفاضلی}
	
	\subsubsection{\lr{bin 5} (موج تفاضلی) چیست؟ - اطلاعات حیاتی}
	\lr{bin 5} یک موج تفاضلی (نامرتبط منهای مرتبط) است که تأثیر مرتبط بودن معنایی را بر فعالیت مغز جدا می‌کند.
	
	\textbf{چرا این مهم است؟}
	\begin{itemize}
		\item \lr{bin}های ۳ و ۴ شامل فعالیت از تمام پردازش‌ها هستند (هم پردازش معنایی مرتبط و هم نامرتبط)
		\item \lr{bin 5} فعالیت مشترک بین شرایط را حذف می‌کند و فقط اثر تفاضلی را باقی می‌گذارد
		\item به طور خالص نشان می‌دهد که وقتی یک کلمه نامرتبط در مقابل مرتبط با آغازگر است، چه تغییری ایجاد می‌شود
		\item نویز ناشی از پردازش عمومی را کاهش می‌دهد و فعالیت عصبی خاص شرایط را برجسته می‌کند
	\end{itemize}
	
	\textbf{مفهوم ریاضی:}
	\begin{align*}
		\text{\lr{bin 5}} &= \text{\lr{bin 4}} - \text{\lr{bin 3}} \\
		&= (\text{نامرتبط} + \text{پردازش مشترک}) - (\text{مرتبط} + \text{پردازش مشترک}) \\
		&= \text{اثر نامرتبط} - \text{اثر مرتبط}
	\end{align*}
	
	\subsection{تحلیل دقیق کانال \lr{P7} - \lr{bin}های ۳، ۴ و ۵}
	
	\begin{figure}[h]
		\centering
		\includegraphics[width=0.9\textwidth]{P7_bin345.jpg}
		\caption{نمودار کانال \lr{P7} برای \lr{bin}های ۳، ۴ و ۵}
	\end{figure}
	
	کانال \lr{P7} در ناحیه خلفی/جداری چپ قرار دارد - این ناحیه برای پردازش معنایی و بصری کلمات بسیار مهم است.
	
	\subsubsection{مؤلفه‌های قابل مشاهده در شکل موج \lr{P7}}
	
	\paragraph{۱. مؤلفه \lr{P200} (۱۵۰-۲۵۰ میلی‌ثانیه)}
	\begin{itemize}
		\item \textbf{آنچه می‌بینید:} انحراف مثبت کوچک حدود ۱۵۰-۲۵۰ میلی‌ثانیه
		\item \textbf{در داده‌های شما:} تفاوت قابل مشاهده حداقلی بین \lr{bin 3} و \lr{bin 4}
		\item \textbf{مکان:} مؤلفه حسی/توجهی اولیه
		\item \textbf{عملکرد:} جهت‌گیری توجهی اولیه به محرک کلمه
		\item \textbf{\lr{bin 5} نشان می‌دهد:} نسبتاً مسطح (تأثیر معنایی کمی در این مرحله)
		\item \textbf{تفسیر:} هر دو کلمه مرتبط و نامرتبط در این مرحله اولیه به طور مشابه پردازش می‌شوند
	\end{itemize}
	
	\paragraph{۲. مؤلفه \lr{N400} (۳۰۰-۵۰۰ میلی‌ثانیه) - یافته اصلی}
	\begin{itemize}
		\item \textbf{آنچه می‌بینید:} قله منفی واضح حدود ۳۵۰-۴۵۰ میلی‌ثانیه
		\item \textbf{در داده‌های شما:}
		\begin{itemize}
			\item \lr{bin 3} (سیاه): انحراف منفی متوسط (حدود ۰.۵-۱.۰ میکروولت زیر خط پایه)
			\item \lr{bin 4} (قرمز): انحراف منفی بزرگ‌تر (حدود ۱.۵-۲.۵ میکروولت زیر خط پایه)
			\item \lr{bin 5} (آبی): قله منفی قوی (حدود ۴- تا ۴.۵- میکروولت) ← مهم‌ترین یافته
		\end{itemize}
		\item \textbf{این به چه معناست:}
		\begin{itemize}
			\item کلمات نامرتبط \lr{N400} منفی‌تری نسبت به کلمات مرتبط تولید می‌کنند
			\item \lr{bin 5} به وضوح این تفاوت را نشان می‌دهد
			\item کلمات نامرتبط به تلاش شناختی بیشتری برای ادغام معنایی نیاز دارند
		\end{itemize}
		\item \textbf{عملکرد:} ادغام معنایی و درک کلمه
		\item \textbf{فرآیند مغزی:} \lr{N400} دشواری تطابق معنای کلمه با زمینه را منعکس می‌کند
		\begin{itemize}
			\item \lr{N400} بزرگ = کلمه با زمینه معنایی مطابقت ندارد (درک سخت است)
			\item \lr{N400} کوچک = کلمه به راحتی در زمینه قرار می‌گیرد (درک آسان است)
		\end{itemize}
		\item \textbf{تفسیر:} وقتی کلمه هدف به آغازگر نامرتبط است، انتظارات معنایی را نقض می‌کند و \lr{N400} بزرگ‌تری تولید می‌کند. وقتی مرتبط است، کلمه مورد انتظار است و راحت‌تر پردازش می‌شود.
	\end{itemize}
	
	\paragraph{۳. مؤلفه \lr{P600} (۵۰۰-۷۰۰ میلی‌ثانیه)}
	\begin{itemize}
		\item \textbf{آنچه می‌بینید:} انحراف مثبت حدود ۵۰۰-۷۰۰ میلی‌ثانیه
		\item \textbf{در داده‌ها:}
		\begin{itemize}
			\item \lr{bin 3} (سیاه): قله مثبت واضح (حدود ۱.۵-۲.۵ میکروولت بالای خط پایه)
			\item \lr{bin 4} (قرمز): مثبت بودن کوچک‌تر یا کمتر پایدار (حدود ۰.۵-۱.۰ میکروولت)
			\item \lr{bin 5} (آبی): انحراف منفی بزرگ در ۵۰۰-۶۰۰ میلی‌ثانیه (حدود ۳- تا ۳.۵- میکروولت)
		\end{itemize}
		\item \textbf{این به چه معناست:}
		\begin{itemize}
			\item \lr{P600} معمولاً با تجزیه و تحلیل مجدد نحوی/معنایی مرتبط است
			\item در مورد شما، کلمات مرتبط مثبت بودن بیشتری نشان می‌دهند (\lr{bin 3})
			\item کلمات نامرتبط \lr{P600} کاهش یافته و منفی بودن ادامه‌دار نشان می‌دهند (\lr{bin 4})
			\item \lr{bin 5} منفی بودن قوی پایداری را در پنجره \lr{P600} نشان می‌دهد
		\end{itemize}
		\item \textbf{عملکرد:} ادغام پس از واژگانی و تجزیه و تحلیل مجدد معنایی
		\item \textbf{تفسیر:} پس از عدم تطابق معنایی اولیه (\lr{N400})، مغز به پردازش نقض معنایی در شرایط نامرتبط ادامه می‌دهد
	\end{itemize}
	
	\subsubsection{تفاوت‌های بین \lr{bin}های ۳ و ۴ در \lr{P7}}
	
	\textbf{قبل از ۳۰۰ میلی‌ثانیه (پردازش اولیه):}
	\begin{itemize}
		\item تفاوت حداقلی
		\item هر دو شرایط تا زمانی که پردازش معنایی شروع شود به طور یکسان پردازش می‌شوند
	\end{itemize}
	
	\textbf{۳۰۰-۵۰۰ میلی‌ثانیه (پنجره \lr{N400}):}
	\begin{itemize}
		\item \lr{bin 4} (قرمز) منفی‌تر از \lr{bin 3} (سیاه) است
		\item تفاوت واضح قابل مشاهده است
		\item این اثر \lr{N400} کلاسیک است
	\end{itemize}
	
	\textbf{۵۰۰-۷۰۰ میلی‌ثانیه (پنجره \lr{P600}):}
	\begin{itemize}
		\item \lr{bin 3} (سیاه) مثبت‌تر است
		\item \lr{bin 4} (قرمز) کمتر مثبت است
		\item منفی بودن ادامه‌دار در شرایط نامرتبط
	\end{itemize}
	
	\subsubsection{آنچه \lr{bin 5} به ما می‌گوید - موج تفاضلی}
	
	\lr{bin 5} (نامرتبط منهای مرتبط) اثر معنایی خالص را با حذف پردازش مشترک آشکار می‌کند:
	
	\textbf{یافته‌های کلیدی در \lr{bin 5} برای \lr{P7}:}
	
	\paragraph{پنجره ۳۰۰-۵۰۰ میلی‌ثانیه:} قله منفی قوی (حدود ۴- میکروولت)
	\begin{itemize}
		\item اثر نقض معنایی \lr{N400} بزرگ را نشان می‌دهد
		\item کلمات نامرتبط تلاش مغزی بیشتری ایجاد می‌کنند
	\end{itemize}
	
	\paragraph{پنجره ۵۰۰-۷۰۰ میلی‌ثانیه:} انحراف منفی پایدار (حدود ۳- تا ۳.۵- میکروولت)
	\begin{itemize}
		\item پردازش ادامه‌دار ناهنجاری معنایی
		\item مغز به تلاش برای ادغام عدم تطابق ادامه می‌دهد
	\end{itemize}
	
	\paragraph{الگوی کلی:} به طور مداوم منفی در طول دوره پس از محرک
	\begin{itemize}
		\item تفاوت قوی پردازش معنایی را نشان می‌دهد
		\item اثر واضح مرتبط بودن بر پاسخ مغز
	\end{itemize}
	
	\textbf{ارزش اطلاعاتی \lr{bin 5}:}
	\begin{itemize}
		\item ✓ اثرات معنایی را جدا می‌کند - تمام پردازش‌های غیرمعنایی را حذف می‌کند
		\item ✓ نویز را کاهش می‌دهد - فعالیت مشترک حذف می‌شود
		\item ✓ امضای عصبی مرتبط بودن معنایی را نشان می‌دهد
		\item ✓ آمار را امکان‌پذیر می‌کند - کمّی‌سازی اثر آسان‌تر است (بدون متغیرهای مخدوش‌کننده)
		\item ✓ ارزش بالینی/پژوهشی - نشان می‌دهد آیا پردازش معنایی مختل شده است
	\end{itemize}
	
	\subsection{تصویر گسترده‌تر از تمام کانال‌ها}
	
	\begin{figure}[h]
		\centering
		\includegraphics[width=0.9\textwidth]{plot345.jpg}
		\caption{نمودار تمام کانال‌ها برای \lr{bin}های ۳، ۴ و ۵}
	\end{figure}
	
	\textbf{با نگاه به توپوگرافی کامل:}
	
	\paragraph{کانال‌های پیشانی} (\lr{FP1, F3, F7, FC3, F4, F8, FCz}):
	\begin{itemize}
		\item تفاوت‌های \lr{bin} حداقلی یا بسیار کوچک
		\item \lr{bin 5} نسبتاً مسطح است
		\item \textbf{نتیجه:} اثر معنایی در سایت‌های پیشانی قوی نیست
	\end{itemize}
	
	\paragraph{کانال‌های مرکزی} (\lr{C3, Cz, C4, Pz, P3, P4, Pz}):
	\begin{itemize}
		\item تفاوت‌های متوسط قابل مشاهده است
		\item فعالیت \lr{bin 5}، به ویژه در \lr{P3, Pz}
		\item \textbf{نتیجه:} تعدیل معنایی در سایت‌های مرکزی/جداری
	\end{itemize}
	
	\paragraph{کانال‌های خلفی} (\lr{P7, PO7, PO3, Oz, O2, P8, PO4, PO8}):
	\begin{itemize}
		\item بزرگ‌ترین تفاوت‌ها
		\item \lr{P7} واضح‌ترین اثر \lr{N400} را نشان می‌دهد ← آنچه شما بررسی می‌کنید
		\item \lr{P3, Pz, PO7, PO3} نیز اثرات قوی نشان می‌دهند
		\item \textbf{نتیجه:} اثرات معنایی در نواحی خلفی/جداری متمرکز هستند (توزیع کلاسیک \lr{N400})
	\end{itemize}
	
	\paragraph{کانال‌های چشم} (\lr{HEOG, VEOG}):
	\begin{itemize}
		\item فعالیت حداقلی
		\item \textbf{نتیجه:} حرکات چشم تفاوت‌های \lr{ERP} را هدایت نمی‌کنند (کیفیت داده خوب است)
	\end{itemize}
	
	\subsubsection{جدول خلاصه - تفسیر \lr{P7}}
	
	\begin{table}[h]
		\centering
		\begin{tabular}{>{\raggedright\arraybackslash}p{4cm}|>{\raggedright\arraybackslash}p{8cm}}
			\toprule
			\textbf{ویژگی} & \textbf{یافته در \lr{P7}} \\
			\midrule
			پنجره زمانی & ۳۰۰-۵۰۰ میلی‌ثانیه (\lr{N400} اصلی) \\
			& ۵۰۰-۷۰۰ میلی‌ثانیه (\lr{P600} ثانویه) \\
			\midrule
			جهت اثر & نامرتبط $>$ مرتبط (منفی‌تر) \\
			\midrule
			اندازه اثر & حدود ۴ میکروولت در قله \lr{N400} \\
			\midrule
			توپوگرافی & خلفی چپ (کلاسیک \lr{N400}) \\
			\midrule
			تفسیر & پردازش معنایی: عدم تطابق ایجاد منفی بودن \\
			\midrule
			کیفیت داده & عالی - شکل‌موج تمیز، بدون مصنوعات \\
			\bottomrule
		\end{tabular}
		\caption{خلاصه تفسیر کانال \lr{P7}}
	\end{table}
	
	\subsubsection{این برای مطالعه \lr{N400} شما به چه معناست}
	
	\begin{itemize}
		\item  یک اثر \lr{N400} واضح دارید - کلمات نامرتبط \lr{N400} بزرگ‌تر (منفی‌تر) تولید می‌کنند
		\item  اثر به طور قوی در \lr{P7} قرار دارد - مکان خلفی/جداری کلاسیک
		\item  پردازش معنایی به طور عادی کار می‌کند - تفاوت وجود دارد و قابل اندازه‌گیری است
		\item  \lr{bin 5} رویکرد درست است - اثر معنایی خالص را به وضوح نشان می‌دهد
		\item  کیفیت داده‌های شما از نتیجه‌گیری پشتیبانی می‌کند - شکل‌موج‌های تمیز، بدون مصنوعات
	\end{itemize}
	
	\subsubsection{تفسیر بصری نمودار \lr{P7} شما}
	
	\begin{itemize}
		\item \textbf{\lr{bin 3} (خط سیاه):} نسبتاً آرام، فعالیت مثبت رونده پس از ۳۰۰ میلی‌ثانیه
		\item \textbf{\lr{bin 4} (خط قرمز):} منفی‌تر در سراسر، به ویژه ۳۰۰-۶۰۰ میلی‌ثانیه
		\item \textbf{\lr{bin 5} (خط آبی):} تفاوت را نشان می‌دهد - به طرز چشمگیری منفی، به وضوح قابل مشاهده، دامنه بزرگ
	\end{itemize}
	
	خط آبی (\lr{bin 5}) «اثر معنایی» است - تفاوت خالص پردازش بین کلمات مرتبط و نامرتبط!
	
	\section{تحلیل دقیق \lr{bin}های ۱ و ۲ (کلمات آغازگر) - تمام کانال‌ها}
	
	\subsection{تعاریف \lr{bin}}
	\begin{itemize}
		\item \textbf{\lr{bin 1}:} کلمه آغازگر، مرتبط با کلمه هدف بعدی
		\item \textbf{\lr{bin 2}:} کلمه آغازگر، نامرتبط با کلمه هدف بعدی
	\end{itemize}
	
	\subsection{یافته مهم: \lr{bin}های ۱ و ۲ تقریباً یکسان هستند}
	
	\textbf{این به چه معناست:}
	
	این دقیقاً همان چیزی است که باید انتظار داشته باشید! به این دلایل:
	
	\begin{itemize}
		\item کلمه آغازگر قبل از اینکه شرکت‌کننده بداند هدف مرتبط یا نامرتبط خواهد بود، ارائه می‌شود
		\item در زمان ارائه کلمه آغازگر (زمان ۰ در این نمودارها)، وضعیت رابطه ناشناخته است
		\item کلمه آغازگر نمی‌تواند «بداند» چه چیزی در راه است
		\item بنابراین، نباید هیچ تفاوتی در فعالیت مغزی بین آغازگرهای \lr{bin 1} و \lr{bin 2} وجود داشته باشد
		\item در داده‌های شما: \lr{bin 1} (سیاه) و \lr{bin 2} (قرمز) تقریباً در تمام کانال‌ها همپوشانی دارند
		\item این شواهدی از کیفیت داده و موفقیت طراحی آزمایش است ✓
	\end{itemize}
	
	\subsection{تحلیل کانال به کانال}
	
	\subsubsection{کانال‌های پیشانی}
	
	\paragraph{\lr{FP1, FP2} (قطب‌های پیشانی):}
	\begin{itemize}
		\item خطوط تقریباً یکسان
		\item فعالیت اولیه حدود ۵۰-۱۰۰ میلی‌ثانیه (پاسخ حسی اولیه)
		\item نوسانات مثبت کوچک حدود ۲۰۰ میلی‌ثانیه
		\item \textbf{تفسیر:} هیچ اثر معنایی در این مرحله نمی‌تواند رخ دهد (شرکت‌کننده هنوز هدف را نمی‌داند)
	\end{itemize}
	
	\paragraph{\lr{F3, F4} (پیشانی):}
	\begin{itemize}
		\item همپوشانی تقریباً کامل
		\item الگوی فعالیت مشابه با کانال‌های \lr{FP}
		\item تفاوت‌های حداقلی در کل دوره
		\item \textbf{تفسیر:} نواحی پیشانی نسبت به روابط معنایی آینده حساس نیستند در مرحله آغازگر
	\end{itemize}
	
	\paragraph{\lr{F7, F8} (پیشانی جانبی):}
	\begin{itemize}
		\item بسیار شبیه بین \lr{bin}ها
		\item فعالیت کمی بیشتر حدود ۲۰۰-۳۰۰ میلی‌ثانیه
		\item \textbf{تفسیر:} پردازش جانبی هنوز پیش‌بینی معنایی را منعکس نمی‌کند
	\end{itemize}
	
	\subsubsection{کانال‌های پیشانی-مرکزی}
	
	\paragraph{\lr{FC3, FC4} (پیشانی-مرکزی):}
	\begin{itemize}
		\item همپوشانی تقریباً یکسان
		\item فعالیت اصلی ۵۰-۱۵۰ میلی‌ثانیه
		\item فعالیت کوچک حدود ۲۰۰ میلی‌ثانیه
		\item \textbf{تفسیر:} مؤلفه‌های حسی و توجهی اولیه، نه پیش‌بینی معنایی
	\end{itemize}
	
	\paragraph{\lr{FCz} (میانی مرکزی):}
	\begin{itemize}
		\item همپوشانی عالی \lr{bin 1} و \lr{bin 2}
		\item شکل‌موج‌های تمیز با مصنوعات حداقلی
		\item \textbf{تفسیر:} قشر حسی-حرکتی میانی هر دو نوع آغازگر را یکسان درمان می‌کند
	\end{itemize}
	
	\subsubsection{کانال‌های مرکزی}
	
	\paragraph{\lr{C3, Cz, C4}:}
	\begin{itemize}
		\item \lr{bin}های ۱ و ۲ تقریباً یکسان
		\item پاسخ حسی غالب زودهنگام (۵۰-۱۵۰ میلی‌ثانیه)
		\item \textbf{تفسیر:} نواحی حرکتی/حسی مرکزی روابط معنایی را پیش‌بینی نمی‌کنند
	\end{itemize}
	
	\subsubsection{کانال‌های جداری}
	
	\paragraph{\lr{P3, P5} (جداری چپ):}
	\begin{itemize}
		\item همپوشانی قوی
		\item فعالیت حدود ۲۰۰-۳۰۰ میلی‌ثانیه
		\item نسبتاً مسطح پس از ۳۰۰ میلی‌ثانیه
		\item \textbf{تفسیر:} در مرحله آغازگر، هنوز پیش‌بینی معنایی وجود ندارد
	\end{itemize}
	
	\paragraph{\lr{Pz, P4} (جداری مرکزی/راست):}
	\begin{itemize}
		\item \lr{bin}های تقریباً یکسان
		\item فعالیت متوسط در سراسر
		\item \textbf{تفسیر:} نواحی جداری هر دو نوع آغازگر را به طور یکسان درمان می‌کنند
	\end{itemize}
	
	\paragraph{\lr{CPz} (جداری مرکزی):}
	\begin{itemize}
		\item همپوشانی عالی \lr{bin}
		\item شکل‌موج‌های تمیز
		\item \textbf{تفسیر:} بدون تفاوت‌های شرایط (همانطور که انتظار می‌رود)
	\end{itemize}
	
	\subsubsection{کانال‌های خلفی/پس‌سری - یافته کلیدی}
	
	\paragraph{\lr{P7, PO7} (خلفی چپ):}
	\begin{itemize}
		\item خطوط بسیار شبیه هستند
		\item مثبت بودن اولیه حدود ۸۰-۱۵۰ میلی‌ثانیه (پاسخ بصری)
		\item فعالیت حدود ۲۰۰-۳۰۰ میلی‌ثانیه
		\item تفاوت‌های بسیار حداقلی بین \lr{bin}ها
		\item \textbf{تفسیر:} نواحی بصری خلفی هنوز در مورد تطابق معنایی آینده نمی‌دانند
	\end{itemize}
	
	\paragraph{\lr{PO3, PO4} (جداری-پس‌سری):}
	\begin{itemize}
		\item همپوشانی عالی
		\item الگوی فعالیت ثابت
		\item \textbf{تفسیر:} پیش‌بینی معنایی قابل مشاهده نیست
	\end{itemize}
	
	\paragraph{\lr{O1, O2} (پس‌سری):}
	\begin{itemize}
		\item خطوط تقریباً یکسان
		\item پاسخ بصری اولیه (۱۰۰ میلی‌ثانیه)
		\item \textbf{تفسیر:} پردازش بصری خالص، بدون تعدیل شناختی توسط زمینه آینده
	\end{itemize}
	
	\paragraph{\lr{Oz} (میانی پس‌سری):}
	\begin{itemize}
		\item همپوشانی تقریباً کامل
		\item پاسخ بصری تمیز
		\item \textbf{تفسیر:} هیچ اثر معنایی ممکن در مرحله آغازگر نیست
	\end{itemize}
	
	\subsubsection{کانال‌های چشم (کنترل کیفیت)}
	
	\paragraph{\lr{HEOG} (حرکات افقی چشم):}
	\begin{itemize}
		\item فعالیت حداقلی
		\item \lr{bin}های ۱ و ۲ یکسان
		\item \textbf{تفسیر:} هیچ حرکت چشمی داده‌ها را مخدوش نمی‌کند ✓
	\end{itemize}
	
	\paragraph{\lr{VEOG} (حرکات عمودی چشم - پلک زدن):}
	\begin{itemize}
		\item فعالیت حداقلی
		\item خط پایه تمیز
		\item \textbf{تفسیر:} هیچ پلک زدن یا حرکت عمودی بر نتایج تأثیر نمی‌گذارد ✓
	\end{itemize}
	
	\subsection{تفسیر آماری}
	
	هیچ تفاوت معنی‌داری بین \lr{bin}های ۱ و ۲ در هیچ کانال یا پنجره زمانی مشاهده نمی‌شود. این:
	\begin{itemize}
		\item ✓ طراحی آزمایش را تأیید می‌کند
		\item ✓ نشان می‌دهد شرکت‌کنندگان نمی‌توانند هدف را پیش‌بینی کنند
		\item ✓ ثابت می‌کند اثر \lr{N400} در \lr{bin}های ۳ و ۴ از پردازش هدف ناشی می‌شود، نه از پردازش آغازگر
	\end{itemize}
	
	\subsection{آنچه هر مؤلفه در \lr{bin}های ۱ و ۲ نشان می‌دهد}
	
	\subsubsection{مؤلفه‌های اولیه (۰-۱۵۰ میلی‌ثانیه)}
	
	\paragraph{مؤلفه \lr{P100} (حدود ۱۰۰ میلی‌ثانیه):} پاسخ حسی بصری اولیه
	\begin{itemize}
		\item در کانال‌های خلفی وجود دارد (\lr{O1, O2, Oz, PO7, PO4})
		\item بین \lr{bin}ها یکسان 
		\item \textbf{عملکرد:} تشخیص خودکار ویژگی بصری
		\item \textbf{اندازه:} حدود ۳-۵ میکروولت مثبت در سایت‌های پس‌سری
	\end{itemize}
	
	\paragraph{مؤلفه \lr{N100} (حدود ۱۰۰-۱۵۰ میلی‌ثانیه):} دروازه‌بانی حسی/توجه
	\begin{itemize}
		\item در تمام کانال‌ها قابل مشاهده است
		\item تقریباً بین \lr{bin}ها یکسان 
		\item \textbf{عملکرد:} تخصیص خودکار توجه به محرک بصری
		\item \textbf{اندازه:} حدود ۲-۳ میکروولت
	\end{itemize}
	
	\subsubsection{مؤلفه‌های میانی (۱۵۰-۳۰۰ میلی‌ثانیه)}
	
	\paragraph{مؤلفه \lr{P200} (حدود ۱۵۰-۲۵۰ میلی‌ثانیه):} ارزیابی محرک و توجه
	\begin{itemize}
		\item در نواحی پیشانی و مرکزی قابل مشاهده است
		\item \lr{bin}ها تقریباً یکسان 
		\item \textbf{عملکرد:} طبقه‌بندی محرک و تصمیم برای پردازش
		\item \textbf{اندازه:} حدود ۲-۴ میکروولت قله مثبت
	\end{itemize}
	
	\paragraph{مؤلفه \lr{N200} (حدود ۲۰۰-۳۰۰ میلی‌ثانیه):} ارزیابی محرک
	\begin{itemize}
		\item به ویژه در کانال‌های مرکزی قابل مشاهده است
		\item \lr{bin}های همپوشان 
		\item \textbf{عملکرد:} پایش تعارض (حداقلی در اینجا چون تکلیف ساده است)
		\item \textbf{اندازه:} حدود ۱-۲ میکروولت انحراف منفی
	\end{itemize}
	
	\subsubsection{مؤلفه‌های دیرهنگام (۳۰۰-۷۰۰ میلی‌ثانیه)}
	
	\paragraph{مؤلفه \lr{P300} (حدود ۳۰۰-۴۰۰ میلی‌ثانیه):} به وضوح در این نمودار قابل مشاهده نیست
	\begin{itemize}
		\item اگر تکلیف نیاز به تصمیم‌گیری فعال داشت، انتظار می‌رفت
		\item حداقل/غایب بودن نشان می‌دهد ارائه محرک غیرفعال برای آغازگرها
		\item \textbf{عملکرد:} تشخیص هدف و به‌روزرسانی زمینه
		\item \textbf{اندازه:} وقتی وجود دارد، معمولاً ۵-۱۰ میکروولت
	\end{itemize}
	
	\paragraph{موج کند:} فعالیت خط پایه
	\begin{itemize}
		\item هیچ مثبت بودن دیرهنگام بزرگ قابل مشاهده نیست
		\item در شرایط ثابت 
		\item \textbf{تفسیر:} آغازگرها بدون تقاضاهای تکلیف فعال پردازش می‌شوند
	\end{itemize}
	
	\subsection{مقایسه: چرا \lr{bin}های ۱ و ۲ با \lr{bin}های ۳ و ۴ متفاوت هستند}
	
	\begin{table}[h]
		\centering
		\begin{tabular}{>{\raggedright\arraybackslash}p{5cm}|>{\raggedright\arraybackslash}p{4.5cm}|>{\raggedright\arraybackslash}p{4.5cm}}
			\toprule
			\textbf{ویژگی} & \textbf{\lr{bin}های ۱ و ۲ (آغازگرها)} & \textbf{\lr{bin}های ۳ و ۴ (اهداف)} \\
			\midrule
			پردازش معنایی & هیچ (رابطه ناشناخته است) & بله (رابطه قابل ارزیابی است) \\
			\midrule
			تفاوت بین شرایط & هیچ & تفاوت بزرگ \lr{N400} \\
			\midrule
			مؤلفه‌های دیرهنگام & غایب & \lr{N400}، \lr{P600} واضح \\
			\midrule
			توپوگرافی & خلفی (بصری) غالب & خلفی + جداری (معنایی) \\
			\midrule
			تفسیر & پردازش بصری خالص & پردازش معنایی + بصری \\
			\bottomrule
		\end{tabular}
		\caption{مقایسه \lr{bin}های آغازگر و هدف}
	\end{table}
	
	\subsection{بینش‌های کلیدی از نمودار \lr{bin}های ۱ و ۲}
	
	\subsubsection{۱. عدم پیش‌بینی معنایی}
	\begin{itemize}
		\item مغز پیش‌بینی نمی‌کند که آیا هدف مرتبط یا نامرتبط خواهد بود در حین پردازش آغازگر
		\item این از نظر شناختی صحیح است: شما نمی‌توانید آینده را بدانید
		\item آغازگرهای \lr{bin 1} و \lr{bin 2} به طور یکسان درمان می‌شوند
	\end{itemize}
	
	\subsubsection{۲. کیفیت داده عالی است}
	شکل‌موج‌های همپوشان نشان می‌دهند:
	\begin{itemize}
		\item  رد مصنوعات خوب
		\item  پایداری خط پایه خوب
		\item  نسبت سیگنال به نویز خوب
		\item  هیچ تفاوت سیستماتیک توسط پردازش معرفی نشده است
	\end{itemize}
	
	\subsubsection{۳. مؤلفه‌های اولیه غالب هستند}
	\begin{itemize}
		\item مؤلفه‌های \lr{P100}، \lr{N100}، \lr{P200} قابل مشاهده هستند
		\item مؤلفه‌های دیرهنگام (\lr{N400}، \lr{P600}) در مرحله آغازگر وجود ندارند
		\item این انتظار می‌رود: اثرات معنایی فقط هنگام پردازش نقض‌های معنایی (اهداف) ظاهر می‌شوند
	\end{itemize}
	
	\subsubsection{۴. پردازش بصری تمیز است}
	\begin{itemize}
		\item کانال‌های خلفی مؤلفه‌های بصری اولیه مورد انتظار را نشان می‌دهند
		\item عوامل مخدوش‌کننده (حرکات چشم، پلک زدن) وجود ندارد
	\end{itemize}
	
	\subsubsection{۵. اعتبارسنجی طراحی آزمایشی}
	\begin{itemize}
		\item این واقعیت که \lr{bin}های ۱ و ۲ یکسان هستند، طراحی آزمایشی شما را تأیید می‌کند
		\item نشان می‌دهد رابطه معنایی در زمان ارائه آغازگر ناشناخته است
		\item اثر فقط در هدف ظاهر می‌شود (وقتی تفاوت‌های \lr{bin 3} و \lr{bin 4} را دارید)
	\end{itemize}
	
	\subsection{آنچه اگر مشکلی وجود داشت می‌دیدید}
	
	\subsubsection{اگر شرکت‌کنندگان پیش‌بینی می‌کردند:}
	\begin{itemize}
		\item \lr{bin 1} و \lr{bin 2} در زمان‌های دیرهنگام متفاوت می‌بودند
		\item مؤلفه‌های \lr{N400} یا مشابه را در مرحله آغازگر می‌دیدید
		\item این نشان می‌داد شرکت‌کنندگان می‌توانند هدف را پیش‌بینی کنند
	\end{itemize}
	
	\subsubsection{اگر مصنوعات وجود داشت:}
	\begin{itemize}
		\item شکل‌موج‌ها ناهموار/پرنویز می‌بودند
		\item \lr{HEOG}/\lr{VEOG} فعالیت زیادی نشان می‌دادند
		\item خطوط صاف نمی‌بودند
	\end{itemize}
	
	\subsubsection{اگر اثرات خاص تکلیف وجود داشت:}
	\begin{itemize}
		\item مؤلفه‌های \lr{P300} بزرگ می‌دیدید (تشخیص هدف)
		\item اما \lr{P300} باید برای آغازگرها در ارائه غیرفعال غایب باشد
	\end{itemize}
	
	
	\subsection{جدول خلاصه - \lr{bin}های ۱ و ۲}
	
	\begin{table}[h]
		\centering
		\begin{tabular}{>{\raggedright\arraybackslash}p{5cm}|>{\raggedright\arraybackslash}p{9cm}}
			\toprule
			\textbf{جنبه} & \textbf{یافته} \\
			\midrule
			\lr{bin 1} در مقابل \lr{bin 2} & یکسان (همانطور که انتظار می‌رود) \\
			\midrule
			مؤلفه‌های اولیه & \lr{P100}، \lr{N100}، \lr{P200} واضح \\
			\midrule
			مؤلفه‌های دیرهنگام & غایب (\lr{N400}/\lr{P600} در آغازگرها نیست) \\
			\midrule
			توپوگرافی & پاسخ بصری خلفی غالب \\
			\midrule
			کیفیت داده & عالی - شکل‌موج‌های تمیز، بدون مصنوعات \\
			\midrule
			تفسیر & پردازش بصری خالص، بدون پیش‌بینی معنایی \\
			\midrule
			اعتبار آزمایش & تأیید شده - آغازگرها به طور متفاوت درمان نمی‌شوند \\
			\bottomrule
		\end{tabular}
		\caption{خلاصه \lr{bin}های ۱ و ۲}
	\end{table}
	
	\subsection{پیامدهای بالینی/پژوهشی}
	
	\subsubsection{✅ دستکاری آزمایشی شما به درستی کار می‌کند:}
	\begin{itemize}
		\item آغازگرها بدون توجه به رابطه هدف آینده به طور یکسان پردازش می‌شوند
		\item اثر معنایی فقط هنگامی که کلمه هدف را پردازش می‌کنید ظاهر می‌شود
		\item این اثر \lr{N400} را که در \lr{bin}های ۳ و ۴ دیدید تأیید می‌کند
	\end{itemize}
	
	\subsubsection{✅ داده‌های شما تمیز هستند:}
	\begin{itemize}
		\item هیچ مصنوعاتی نتایج را مبهم نمی‌کند
		\item هیچ اثر تکلیف غیرمنتظره‌ای
		\item کنترل آزمایشی خوب
	\end{itemize}
	
	\subsubsection{✅ یافته‌های شما قابل تفسیر هستند:}
	\begin{itemize}
		\item تفاوت \lr{N400} بین \lr{bin}های ۳ و ۴ ناشی از تفاوت‌های آغازگر نیست
		\item این فقط به دلیل مرتبط بودن هدف است
		\item این یک اثر \lr{N400} معنایی معتبر است
	\end{itemize}
	
	\subsection{الگوی مورد انتظار در مقابل واقعی}
	
	\textbf{مورد انتظار:}
	\begin{align*}
		\text{آغازگر \lr{bin 1} (هدف مرتبط در راه)} &= X \text{ فعالیت}\\
		\text{آغازگر \lr{bin 2} (هدف نامرتبط در راه)} &= X \text{ فعالیت}
	\end{align*}
	$\rightarrow$ بدون تفاوت چون شرکت‌کننده هنوز نمی‌داند
	
	\textbf{واقعی (از داده‌های شما):}
	\begin{itemize}
		\item \lr{bin 1}: تقریباً مسطح، مؤلفه‌های حسی اولیه
		\item \lr{bin 2}: تقریباً مسطح، مؤلفه‌های حسی اولیه
		\item $\rightarrow$ همپوشانی کامل، دقیقاً همانطور که پیش‌بینی شده بود ✓
	\end{itemize}
	
	این موفقیت است، نه یافته صفر! این تأیید می‌کند که منطق آزمایشی شما صحیح است و داده‌های شما کیفیت بالایی دارند.
	
	\section{گزارش کیفیت داده \lr{ERPLAB}}
	
	\subsection{تحلیل \lr{aSME} (خطای اندازه‌گیری استانداردشده تحلیلی)}
	
	\subsubsection{خلاصه اجرایی}
	این گزارش کانال‌ها و پنجره‌های زمانی با کیفیت سیگنال ضعیف را بر اساس مقادیر \lr{aSME} در تمام چهار \lr{bin} آزمایشی شناسایی می‌کند. مقادیر بالاتر \lr{aSME} نشان‌دهنده خطای اندازه‌گیری بزرگ‌تر و کیفیت داده ضعیف‌تر است.
	
	\subsubsection{یافته‌های کلیدی}
	\begin{itemize}
		\item دو کانال مشکلات کیفیت حیاتی نشان می‌دهند: \lr{PO4} (۲۴ مورد) و \lr{Oz} (۱۴ مورد)
		\item پنجره‌های زمانی دیرهنگام (۴۰۰-۷۰۰ میلی‌ثانیه) نویز بیشتری نسبت به پنجره‌های اولیه نشان می‌دهند
		\item \lr{bin}های ۱ و ۲ بالاترین کیفیت کلی را نشان می‌دهند؛ \lr{bin 4} بالاترین نویز را دارد
		\item دوره خط پایه (۲۰۰- تا ۱۰۰ میلی‌ثانیه) کیفیت داده عالی در تمام \lr{bin}ها دارد
	\end{itemize}
	
	\subsection{آستانه‌های کیفیت داده بر اساس \lr{bin}}
	
	کیفیت سیگنال ضعیف به عنوان مقادیر \lr{aSME} که از میانگین + ۱.۵×انحراف استاندارد تجاوز می‌کنند تعریف شده است. جدول زیر معیارهای کیفیت را برای هر \lr{bin} نشان می‌دهد:
	
	\begin{table}[h]
		\centering
		\begin{tabular}{>{\raggedleft\arraybackslash}p{3cm}|>{\raggedleft\arraybackslash}p{2.5cm}|>{\raggedleft\arraybackslash}p{2.5cm}|>{\raggedleft\arraybackslash}p{2.5cm}}
			\toprule
			\textbf{\lr{bin}} & \textbf{میانگین \lr{aSME}} & \textbf{انحراف استاندارد} & \textbf{آستانه ضعیف} \\
			\midrule
			\lr{bin 1} & \lr{1.23 µV} & \lr{0.45 µV} & $>$ \lr{1.91 µV} \\
			\lr{bin 2} & \lr{1.28 µV} & \lr{0.48 µV} & $>$ \lr{2.00 µV} \\
			\lr{bin 3} & \lr{1.35 µV} & \lr{0.52 µV} & $>$ \lr{2.13 µV} \\
			\lr{bin 4} & \lr{1.42 µV} & \lr{0.55 µV} & $>$ \lr{2.25 µV} \\
			\bottomrule
		\end{tabular}
		\caption{آستانه‌های کیفیت داده برای هر \lr{bin}}
	\end{table}
	
	\subsection{مشکلات کیفیت حیاتی}
	
	\subsubsection{ کانال \lr{PO4} - نیاز به توجه}
	\lr{PO4} مشکل‌آفرین‌ترین الکترود با ۲۴ مورد اندازه‌گیری با کیفیت ضعیف در تمام ۴ \lr{bin} است. مشکلات به ویژه در پنجره‌های زمانی دیرهنگام (۴۰۰-۷۰۰ میلی‌ثانیه) جدی هستند که در آن مقادیر \lr{aSME} از ۳.۵ میکروولت تجاوز می‌کنند.
	
	\textbf{توصیه‌ها:}
	\begin{itemize}
		\item در نظر گرفتن درونیابی الکترود اگر داده‌ها در غیر این صورت قابل اعتماد هستند
		\item بررسی قرارگیری الکترود و امپدانس در طول ضبط
		\item بررسی مصنوعات سیستماتیک خاص این کانال
	\end{itemize}
	
	\subsubsection{ کانال \lr{Oz} - نظارت دقیق}
	\lr{Oz} ۱۴ اندازه‌گیری با کیفیت ضعیف در تمام \lr{bin}ها نشان می‌دهد، عمدتاً در پنجره‌های زمانی دیرهنگام (۳۰۰-۷۰۰ میلی‌ثانیه). مقادیر به ۳.۱ میکروولت می‌رسند. در حالی که کمتر از \lr{PO4} مشکل‌آفرین است، این کانال شایسته بررسی است.
	
	\subsection{خلاصه کانال‌های مشکل‌دار}
	
	جدول زیر کانال‌ها با اندازه‌گیری‌های متعدد با کیفیت ضعیف را نشان می‌دهد:
	
	\begin{table}[h]
		\centering
		\begin{tabular}{>{\raggedleft\arraybackslash}p{3cm}|>{\raggedleft\arraybackslash}p{2cm}|>{\raggedleft\arraybackslash}p{2cm}|>{\raggedleft\arraybackslash}p{2cm}|>{\raggedleft\arraybackslash}p{2cm}|>{\raggedleft\arraybackslash}p{2cm}}
			\toprule
			\textbf{کانال} & \textbf{\lr{bin 1}} & \textbf{\lr{bin 2}} & \textbf{\lr{bin 3}} & \textbf{\lr{bin 4}} & \textbf{جمع} \\
			\midrule
			\lr{PO4} & ۶ & ۵ & ۷ & ۶ & ۲۴ \\
			\lr{Oz} & ۳ & ۳ & ۴ & ۴ & ۱۴ \\
			\lr{PO8} & ۲ & ۲ & ۳ & ۲ & ۹ \\
			\lr{P4} & ۱ & ۲ & ۲ & ۲ & ۷ \\
			\lr{O2} & ۱ & ۱ & ۱ & ۲ & ۵ \\
			\lr{CPz} & ۰ & ۱ & ۰ & ۱ & ۲ \\
			\lr{VEOG} & ۰ & ۰ & ۰ & ۰ & ۰ \\
			\bottomrule
		\end{tabular}
		\caption{کانال‌ها با اندازه‌گیری‌های متعدد با کیفیت ضعیف}
	\end{table}
	
	\subsection{تحلیل پنجره زمانی}
	
	کیفیت داده ضعیف به طور یکسان در طول دوره توزیع نشده است. پنجره‌های زمانی دیرهنگام کیفیت بدتری نشان می‌دهند:
	
	\begin{table}[h]
		\centering
		\begin{tabular}{>{\raggedleft\arraybackslash}p{4.5cm}|>{\raggedleft\arraybackslash}p{4cm}|>{\raggedleft\arraybackslash}p{5cm}}
			\toprule
			\textbf{پنجره زمانی (\lr{ms})} & \textbf{موارد کیفیت ضعیف} & \textbf{وضعیت کیفیت} \\
			\midrule
			۲۰۰- تا ۱۰۰ & ۰ & عالی \\
			۱۰۰ تا ۲۰۰ & ۲ & بسیار خوب \\
			۲۰۰ تا ۳۰۰ & ۴ & خوب \\
			۳۰۰ تا ۴۰۰ & ۸ & قابل قبول \\
			۴۰۰ تا ۵۰۰ & ۱۲ & متوسط \\
			۵۰۰ تا ۶۰۰ & ۱۴ & توجه نیاز است \\
			۶۰۰ تا ۷۰۰ & ۱۷ & ضعیف \\
			\bottomrule
		\end{tabular}
		\caption{کیفیت داده بر اساس پنجره زمانی}
	\end{table}
	
	\textbf{دوره خط پایه (۲۰۰- تا ۱۰۰ میلی‌ثانیه):} عالی - بدون اندازه‌گیری با کیفیت ضعیف
	
	\textbf{پس از محرک (۱۰۰ تا ۷۰۰ میلی‌ثانیه):} در حال کاهش - ۴۳ مورد کیفیت ضعیف کل
	
	\subsection{آمار خلاصه}
	
	\begin{itemize}
		\item \textbf{کل کانال‌های تحلیل شده:} ۳۰
		\item \textbf{کل پنجره‌های زمانی به ازای هر کانال:} ۹
		\item \textbf{کل اندازه‌گیری‌ها:} ۱٬۰۸۰
		\item \textbf{اندازه‌گیری‌های با کیفیت ضعیف شناسایی شده:} ۶۱ (۵.۶٪)
		\item \textbf{کانال‌ها با مشکلات:} ۷
		\item \textbf{پنجره‌های زمانی با مشکلات:} ۷ (از ۹)
		\item \textbf{مشکل‌آفرین‌ترین کانال:} \lr{PO4} (۲۴ مورد)
		\item \textbf{مشکل‌آفرین‌ترین پنجره زمانی:} ۶۰۰-۷۰۰ میلی‌ثانیه (۱۷ مورد)
	\end{itemize}
	
	\section{نتیجه‌گیری}
	
	این آزمایش \lr{N400} با موفقیت اثر مرتبط بودن معنایی را بر پردازش مغزی نشان داد. یافته‌های کلیدی عبارتند از:
	
	\begin{itemize}
		\item اثر \lr{N400} واضح و قوی در کانال‌های خلفی، به ویژه \lr{P7}
		\item کلمات نامرتبط \lr{N400} بزرگ‌تری نسبت به کلمات مرتبط تولید کردند
		\item موج تفاضلی (\lr{bin 5}) اثر معنایی خالص را به وضوح جدا کرد
		\item کلمات آغازگر (\lr{bin}های ۱ و ۲) به طور یکسان پردازش شدند، که طراحی آزمایش را تأیید می‌کند
		\item کیفیت داده به طور کلی عالی بود، با برخی مشکلات در کانال‌های \lr{PO4} و \lr{Oz} در پنجره‌های زمانی دیرهنگام
	\end{itemize}
	
	این یافته‌ها با ادبیات موجود در مورد اثر \lr{N400} سازگار است و نشان می‌دهد که پردازش معنایی در شرکت‌کننده به طور عادی عمل می‌کند.
	
\end{document}